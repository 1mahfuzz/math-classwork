\documentclass{article}
\usepackage[banglamainfont=Kalpurush, 
            banglattfont=Siyam Rupali
           ]{latexbangla}
           
\title{Math - Lec - 2}
\author{Mahfuz Ahmed }
\date{30 April 2020}

\begin{document}
    \maketitle
    \begin{center}
    	\textbf{Abstract}
    \end{center}
    \begin{center}
*Note that I don't own this exercise in this document. This was originally created
for class 8 of Paradise International School. If you have any question please contact Raju Sir. \\
    \end{center}        
    
 %rule
\large 
    \begin{center}
    	\textbf{গ.সা.গু নির্ণয় এর নিয়ম}
    \end{center}

 প্রথমে রাশি গুলোকে উৎপাদকে বিশ্লেষণ করতে হবে। এরপর দেখতে হবে কোনটি সবগুলো রাশির মধ্যে আছে। 
 যেটা সবগুলোর মধ্যে থাকবে সেইটা হবে গ.সা.গু।
%problem 
\vspace{3mm}
\large
\begin{center}
	\textbf{আনুঃ ৪.৪}
\end{center}

১৭। $x^2-3x$ , $x^2-9$ এবং $x^2-4x+3$ রাশি গুলোর গ.সা.গু নির্ণয় কর।\\ \\   
\hspace*{12mm}প্রথম রাশি\hspace{1mm} = $x^2-3x$\\
\hspace*{2.8cm} = $x(x-3)$\\

\hspace*{7mm}দ্বিতীয় রাশি = $x^2-9$\\
\hspace*{2.8cm} = $x^2-3^2$\\
\hspace*{2.8cm} = $(x+3)(x-3)$\\ 

\hspace*{7mm}তৃতীয় রাশি = $x^2-4x+3$\\
\hspace*{2.8cm} = $x^2-3x-x+3$\\
\hspace*{2.8cm} = $x(x-3)-1(x-3)$\\
\hspace*{2.8cm} = $(x-3)(x-1)$\\ \\

\hspace*{7mm}নির্ণেয় ল.সা.গুঃ\\ 
\hspace*{2.8cm}$x-3$\\

\end{document}